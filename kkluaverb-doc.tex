\documentclass[luatex,fontsize=8pt,paper=b5,twoside]{jlreq}%
\usepackage{amsmath,amssymb}
\usepackage{booktabs,caption, tabularray}
\usepackage{luwa-ul}
\usepackage{KKsymbols}
\usepackage[most]{tcolorbox}
\usepackage{KKluaverb}

%%% fonts (This can be omitted.)
\makeatletter 
\RequirePackage[no-math]{fontspec}
\RequirePackage[no-math,match,scale=1]{luatexja-fontspec}
\RequirePackage[hiragino-pro,deluxe,expert]{luatexja-preset}
\setmainfont{HiraMinPro-W3}[BoldFont=HiraMinPro-W6]
\setmainjfont{HiraMinPro-W3}[BoldFont=HiraMinPro-W6]
\newfontfamily{\sfhira@pre}{HiraKakuPro-W3}[BoldFont=HiraKakuPro-W6]
\newjfontfamily{\sfhiraj@pre}{HiraKakuPro-W3}[BoldFont=HiraKakuPro-W6]
\newfontfamily{\mchira@pre}{HiraMinPro-W3}[BoldFont=HiraMinPro-W6]
\newjfontfamily{\mchiraj@pre}{HiraMinPro-W3}[BoldFont=HiraMinPro-W6]
\newfontfamily{\gthira@pre}{HiraKakuPro-W3}[BoldFont=HiraKakuPro-W6,FontFace={eb}{\shapedefault}{HiraKakuStd-W8}]
\newjfontfamily{\gthiraj@pre}{HiraKakuPro-W3}[BoldFont=HiraKakuPro-W6,FontFace={eb}{\shapedefault}{HiraKakuStd-W8}]
\newfontfamily{\mghira@pre}{HiraMaruPro-W4}
\newjfontfamily{\mghiraj@pre}{HiraMaruPro-W4}
\renewcommand{\sffamily}{\sfhira@pre\sfhiraj@pre}
\renewcommand{\mcfamily}{\mchira@pre\mchiraj@pre}
\renewcommand{\gtfamily}{\gthira@pre\gthiraj@pre}
\renewcommand{\mgfamily}{\mghira@pre\mghiraj@pre}
\makeatother
%%%


%%% mapping
\begin{luacode*}
  KKLuaVerb.presets["TeXStyle"] = {
    map = { 
      DarkCyan = { "{", "}" },              
      OrangeRed = { "[", "]" },
      DeepPink = { "(", ")" },              
      DarkGoldenrod = { "&", "$", "^", "_" },

      CornflowerBlue = { 
        "\\documentclass", "\\usepackage", "\\begin", "\\end", 
        "\\section", "\\subsection", "\\chapter", 
        "\\makeatletter", "\\makeatother", "\\ExplSyntaxOn", "\\ExplSyntaxOff" 
      },
      
      MediumPurple = { 
        "\\def", "\\edef", "\\gdef", "\\xdef", 
        "\\newcommand", "\\renewcommand", "\\providecommand",
        "\\let", "\\setcopy", "\\long", "\\global"
      },

      Magenta = { 
        "\\if", "\\else", "\\fi", "\\ifx", "\\ifdefined", 
        "\\ifnum", "\\ifdim", "\\loop", "\\repeat", "\\noexpand"
      },

      IndianRed = {
        "\\KKlvStart*", "\\KKlvEnd*"
      }
    }, 
    
    options = { 
      word_boundary = false,  
      comment_char = "%",     
      comment_color = "ForestGreen", 
      escape_char = "\\"      
    }
  }
\end{luacode*}

\begin{luacode*}
  KKLuaVerb.presets["LuaStyle"] = {
    map = { 
      Magenta = { 
        "if", "then", "else", "elseif", "end", 
        "for", "while", "repeat", "until", "in",
        "do", "break", "return", "function" 
      },

      ProcessBlue = { "local", "true", "false", "nil" },
      DarkGoldenrod = { 
        "print", "require", "type", "tostring", "tonumber",
        "table", "string", "math", "io", "os", "debug", "coroutine",
        "self" 
      },

      DarkCyan = { "{", "}", "[", "]", "(", ")", "=", "+", "-", "*", "/", "#", "..", "==" }
    },
    options = { 
      word_boundary = true,   
      comment_char = "--",     
      comment_color = "Cerulean", 
      escape_char = "%"       
    }
  }
\end{luacode*}
%%%


%%% hyperref
\usepackage{hyperref} 
\hypersetup{
  luatex, pdfencoding=auto, 
  colorlinks=true,
  linkcolor=black,     
  citecolor=black,     
  urlcolor=DeepSkyBlue3,      
  pdfborder={0 0 0}, 
}
%%%


%%% textboxes
\makeatletter
\newcommand{\ascb@parindent}[1]{%
  \setlength{\parindent}{#1}\relax%
}
\def\ascb@gtfamily{\gtfamily}\def\ascb@zw#1#2{#1\zw}
\DeclareTColorBox{simple}{ o m O{.5} O{} }% 
{empty, left=2mm, right=2mm, top=-1mm, 
attach boxed title to top left={xshift=\ascb@zw{1.2}{11pt}}, 
boxed title style={empty,left=-2mm,right=-2mm}, colframe=black, coltitle=black, coltext=black, breakable, 
before upper={\ascb@parindent{\zw}},%%%
underlay unbroken={\draw[black,line width=#3pt](title.east) -- (title.east-|frame.east) -- (frame.south east) -- (frame.south west) -- (title.west-|frame.west) -- (title.west); },
underlay first={\draw[black,line width=#3pt](title.east) -- (title.east-|frame.east) -- (frame.south east) ;
\draw[black,line width=#3pt] (frame.south west) -- (title.west-|frame.west) -- (title.west); },
underlay middle={\draw[black,line width=#3pt](frame.north east) -- (frame.south east) ;
\draw[black,line width=#3pt](frame.south west) -- (frame.north west) ;},
underlay last={\draw[black,line width=#3pt](frame.north east) -- (frame.south east) -- (frame.south west) -- (frame.north west) ;},
fonttitle=\ascb@gtfamily, IfValueTF={#1}{title=\hspace*{.1em}【#2】〈#1〉\hspace*{.1em}}{title=\hspace*{.1em}【#2】\hspace*{.1em}},#4}
\def\KKfbox@SourceCode@phantom{\vphantom{あ}\vphantom{j}}
\NewTCBListing{SourceCode}{ m m !o !O{DeepSkyBlue3} }{%
  enhanced, colback=black!70, colframe=Snow4,
  toptitle=-1mm, bottomtitle=-1mm,
  righttitle=-1mm, lefttitle=-1mm,
  arc=.5mm, 
  title={\tcbox[on line, arc=.5mm, boxsep=0pt, boxrule=0pt, 
    top=1mm, bottom=0.8mm, left=2mm, right=2.2mm, 
    colback=gray!80, coltext=white]{%
    \raisebox{-.04\zw}{\KKfbox@SourceCode@phantom#1}}%
  },
  fonttitle=\gtfamily\footnotesize, boxrule=0.8pt,
  breakable, before upper={\color{white}}, top=-0.5mm, bottom=-0.5mm,
  after title=\IfNoValueTF{#3}{}{{\hfill%
    \tcbox[on line, arc=.5mm, boxsep=0pt, boxrule=0pt, 
    top=1mm, bottom=0.8mm, left=2mm, right=2.2mm, 
    colback=white!80!black, coltext=#4]{\raisebox{-.04\zw}{\KKfbox@SourceCode@phantom#3}}}%
  },
  listing only,
  listing options={
    language={#2},
    basicstyle=\ttfamily,
    keywordstyle=\ttfamily\color{white},
    stringstyle=\itshape\color{white},
    commentstyle=\small\gtfamily\color{DeepSkyBlue2},
    showspaces=false,showtabs=false,
    breaklines=true,breakindent=0pt,
    showstringspaces=false,
    columns=fullflexible,
    tabsize=2,
    numbers=left,numbersep=1.5pt,
    numberstyle=\scriptsize\gtfamily\color{gray},
  }
}
\NewTColorBox{OutPut}{ m !o !O{DeepSkyBlue3} }{%
  enhanced, colframe=Snow4,
  toptitle=-1mm, bottomtitle=-1mm,
  righttitle=-1mm, lefttitle=-1mm,
  arc=.5mm, colback=white, 
  title={\tcbox[on line, arc=.5mm, boxsep=0pt, boxrule=0pt, 
    top=1mm, bottom=0.8mm, left=2mm, right=2.2mm, 
    colback=gray!40, coltext=DeepSkyBlue3]{\raisebox{-.04\zw}{\KKfbox@SourceCode@phantom#1}}%
  },
  fonttitle=\gtfamily\footnotesize, boxrule=0.8pt,
  breakable, top=-0.5mm, bottom=-0.5mm,
  after title=\IfNoValueTF{#2}{}{{\hfill%
    \tcbox[on line, arc=.5mm, boxsep=0pt, boxrule=0pt, 
    top=1mm, bottom=0.8mm, left=2mm, right=2.2mm, 
    colback=white!80!black, coltext=#3]{\raisebox{-.04\zw}{\KKfbox@SourceCode@phantom#2}}}%
  }, 
  bottom=2mm, top=2mm, 
}
\makeatother
%%%


%%% titlepage
\title{\texttt{KKluaverb} Package Documentation}
\author{Kosei Kawaguchi a.k.a. KKTeX}
\date{Version 2.0.1 (2026/01/16)}
%%%

\begin{document}
\KKvUsePreset{TeXStyle}

%%% title and toc
\begin{titlepage}
  \maketitle
\end{titlepage}
\newpage
\tableofcontents
\newpage
%%%

\section{Outline}
The \KKverb|KKluaverb| is a LaTeX package which provides a Lua-enhanced \KKverb|\verb| command, \KKverb|\KKverb|. \thinHighLight{It can be used in the table of contents, index, underline commands (e.g., \KKverb|\underLineKK| provided by the luwa-ul package), \KKverb|tblr| environment and so on.}

\section{Acknowledgements / Credit}
In devoloping this package, I made use of an algorithm which is used in ``bxrawstr'' (by Takayuki YATO)\footnote{%
package: \url{https://gist.github.com/zr-tex8r/c7901658a866adfcd3cd66b6dfa86997}\par
article: \url{https://zrbabbler.hatenablog.com/entry/20181222/1545495849}}.

\section{Basic Usage of Lua-enhanced \KKverb|\verb|}
\subsection{Fundamental Behavior}
As mentioned in the ``Outline'' section, this package mainly provides enhanced \KKverb|\verb| command. The usage is not much different from the normal one.

\KKvScanOff
\begin{SourceCode}{Input}{TeX}
  \KKverb|\def\TEST{Test Text.}|
\end{SourceCode}
\KKvScanOn
\begin{OutPut}{Output}
  \KKverb|\def\TEST{Test Text.}|
\end{OutPut}

In the argument of the command, any linebreaks and blank lines are ignored. However, every space are preserved and rendered exactly as it appears.

\subsection{In Verbatim Environments}
When this package is loaded, the Lua-scanning which enables \KKverb|\KKverb| to detokenize its argument is activated.
However, when \KKverb|\KKverb| used in verbatim environments such as ``lstlistings'' and ``tcblisting'', the argument cannot be decoded properly.
\namiKK{In such cases, you should use \KKverb|\KKvScanOff| which deactivates the scanning process used internally in \KKverb|KKverb.lua|.} Of course, \KKverb|\KKvScanOn| re-activates the process.

\subsection{Optional Usage}
In addition to the main function of \KKverb|\KKverb|, some optional funcions are provided.

\KKverb|\KKvOpChange| can change font and color of the output of \KKverb|\KKverb|. Also, it can activate or deactivate the scanner.
The usage is as follows:

\KKvScanOff
\begin{SourceCode}{Input}{TeX}
  {\KKvOpChange{color=blue, font=\gtfamily, enabled=true}%
  \KKverb|\def\TEST{Test Text.}|}
\end{SourceCode}
\KKvScanOn
\begin{OutPut}{Output}
  {\KKvOpChange{color=blue, font=\gtfamily, enabled=true}%
  \KKverb|\def\TEST{Test Text.}|}
\end{OutPut}

\bigskip\begin{center}
  \captionof{table}{Default Values of \KKverb|\KKvOpChange|}
  \begin{tblr}{
    colspec = {l l l},
    hline{1,Z} = {0.8pt}, 
    hline{2} = {.4pt}, 
    column{1} = {font=\ttfamily}, 
    row{1} = {font=\bfseries, bg=gray9}, 
  }
    Key & Default Value & Description \\
    font & \KKverb|\ttfamily| & Sets the font for the verbatim text. \\
    color & \KKverb|color| & Sets the text color (requires xcolor). \\
    enabled & \KKverb|true| & Activates or deactivates the scanning engine. \\
  \end{tblr}
\end{center}\bigskip

Additionally, the \KKverb|enabled| option functions the same as \KKverb|\KKvScanOn| and \KKverb|\KKvScanOff|.

\subsection{Behavior in TOC, Index etc.}
One of the strong points of \KKverb|\KKverb| is that it can be used in table of contents, index, and footnotes and more. 
While the traditional \KKverb|\verb| command always causes an error in such contexts, \KKverb|\KKverb| does not. This is achieved through the following logic, using the TOC as an example.

When the TeX system generates the TOC, it produces an auxiliary file with the extension \KKverb|.toc|. In this file, entry data is typically stored in an expanded state. 

However, this package requires the raw text data to be passed unexpanded. This is because the text must be intercepted and processed by \KKverb|process_input_buffer| before TeX's tokenizer converts it into tokens. To resolve this, I implemented a mechanism that automatically prepends the \KKverb|\unexpanded| command to the starter flag when writing to auxiliary files: 

\KKcodeS+
% In TOC
\noexpand\KKlvStart*<encoded tests>\noexpand\KKlvEnd*%
\KKcodeE

Then, the content of the \KKverb|.toc| file is processed by TeX in the usual way subsequently, because the \KKverb|\unexpanded| is no longer present.

\section{Basic Usage of Lua-enhanced \KKverb|lstlisting|}
\subsection{Fundamental Behavior}
\KKvScanOff
\verb|\KKcodeS| and \verb|\KKcodeE| provide an environment-like output which is very close to the \verb|lstlisting|.
The usage is very simple. The code between \verb|\KKcodeS| and \verb|\KKcodeE| is rendered as raw texts.

If you want to attach linenumbers to the left of the each line, \verb|\KKcodeS+| will meet the request. Please note that you do not need to type \verb|\KKcodeE+|. \verb|\KKcodeS+| and \verb|\KKcodeE| is the proper pair.

\begin{SourceCode}{Input}{TeX}
  \KKcodeS+
  % A sample command
  \long\def\myprog#1{%
    \ifx#1\empty
      \relax
    \else
      \message{Processing...}%
      #1
    \fi
  }
  \KKcodeE
\end{SourceCode}
\KKvScanOn
\begin{OutPut}{Output}
\KKcodeS+
% A sample command
\long\def\myprog#1{%
  \ifx#1\empty
    \relax
  \else
    \message{Processing...}%
    #1
  \fi
}
\KKcodeE
\end{OutPut}

When you use this Lua-based environment, please keep the following points in mind:

\begin{itemize}
  \item All indents and spaces are displayed.
  \item Line breaks and blank lines are rendered exactly as they appear in the source.
\end{itemize}

\subsection{Additional Description}
\KKvScanOff
Some corner cases should be payed attention. 

First, you cannot use \verb|\KKcodeS| and \verb|\KKcodeE| in the argument of \verb|\KKverb| command. When the sccaner is activated, \verb|\KKcodeS| forcibly starts the encoding system, even it is wrapped by \verb|\KKverb|. 

Any text on the same line that falls "inside" the \verb|\KKcodeS| and \verb|\KKcodeE| boundaries is ignored, while text "outside" them is preserved. The following example will illustrate the behavior.

\begin{SourceCode}{Input}{TeX}
  Not ignored. \KKcodeS+ Ignored.
  % Contents
  Ignored. \KKcodeE Not ignored.
\end{SourceCode}
\KKvScanOn
\begin{OutPut}{Output}
Not ignored. \KKcodeS+ Ignored.
% Contents
Ignored. \KKcodeE Not ignored.
\end{OutPut}

\end{document}

