\documentclass{ltjsarticle}
\usepackage{KKluaverb, luwa-ul}

\begin{luacode*}
  KKLuaVerb.presets["TeXStyle"] = {
    map = { 
      DarkCyan = { "{", "}" },              
      OrangeRed = { "[", "]" },
      DeepPink = { "(", ")" },              
      DarkGoldenrod = { "&", "$", "^", "_" },

      CornflowerBlue = { 
        "\\documentclass", "\\usepackage", "\\begin", "\\end", 
        "\\section", "\\subsection", "\\chapter", 
        "\\makeatletter", "\\makeatother", "\\ExplSyntaxOn", "\\ExplSyntaxOff" 
      },
      
      MediumPurple = { 
        "\\def", "\\edef", "\\gdef", "\\xdef", 
        "\\newcommand", "\\renewcommand", "\\providecommand",
        "\\let", "\\setcopy", "\\long", "\\global"
      },

      Magenta = { 
        "\\if", "\\else", "\\fi", "\\ifx", "\\ifdefined", 
        "\\ifnum", "\\ifdim", "\\loop", "\\repeat"
      },
    }, 
    
    options = { 
      word_boundary = false,  
      comment_char = "%",     
      comment_color = "ForestGreen", 
      escape_char = "\\"      
    }
  }
\end{luacode*}

\begin{luacode*}
  KKLuaVerb.presets["LuaStyle"] = {
    map = { 
      Magenta = { 
        "if", "then", "else", "elseif", "end", 
        "for", "while", "repeat", "until", "in",
        "do", "break", "return", "function" 
      },

      ProcessBlue = { "local", "true", "false", "nil" },
      DarkGoldenrod = { 
        "print", "require", "type", "tostring", "tonumber",
        "table", "string", "math", "io", "os", "debug", "coroutine",
        "self" 
      },

      DarkCyan = { "{", "}", "[", "]", "(", ")", "=", "+", "-", "*", "/", "#", "..", "==" }
    },
    options = { 
      word_boundary = true,   
      comment_char = "--",     
      comment_color = "Cerulean", 
      escape_char = "%"       
    }
  }
\end{luacode*}

\begin{document}
\section{\KKverb|\KKvUsePreset{TeXStyle}| Test}
\KKvUsePreset{TeXStyle}

\underLineKKAuto{Math Mode Test: \KKverb|$a + b = c$|}

無視されない\KKcodeS 無視される
% 1行目のコメント保護とインデントのテスト
\long\def\myprog#1{%
  \ifx#1\empty
    \relax
  \else
    \message{Processing...}%
    #1
  \fi
}
無視される\KKcodeE 無視されない

{\KKvOpChange{font=\sffamily}
\KKcodeS
\let\test\expandafter
\KKcodeE
}

ここでも使えます\footnote{脚注の中でも
\KKcodeS+
  \def\bar{baz} 
  これは\bar です
\KKcodeE 
\noindent と書けるのが KKluaverb の強みです。}。

\begin{enumerate}
\item 箇条書きの項目の中でも:
    \KKcodeS+
    \begin{itemize}
      \item nested item
    \end{itemize}
    \KKcodeE
\end{enumerate}

\KKcodeS+
% ここは行番号が出るはず
\usepackage{xcolor}
\usepackage{KKluaverb}

\begin{document}
  Hello, KKTeX!
\end{document}
\KKcodeE

1行完結のテスト:\KKverb|\long\def\test#1{#1}|です。

記号のテスト:
\KKcodeS
{ [ \ ( \% \$ # & _ ^ ) / ] }
\KKcodeE

\section{\KKverb|\KKvUsePreset{LuaStyle}| Test}
\KKvUsePreset{LuaStyle}

\KKcodeS
  -- test comment
  function KKV.encode_tail(str)
    local s = (str .. "\n"):gsub(ltjflg, "\n")
    return KKV.encode(s)
  end
\KKcodeE

\KKcodeS
-- 制御構文と論理値のテスト
local function check_status(flag)
  if flag == true then
    print("Status is OK! )
  elseif flag == false then
    print("Status is Error... ")
  else
    print("Status is nil.")
  end
end
\KKcodeE

\KKcodeS+
-- ループと標準ライブラリのテスト
local data = { "Apple", "Banana", "Cherry" }

for i, v in ipairs(data) do
  local msg = string.format("Item %d: %s", i, v)
  table.insert(results, msg)
  print(msg)
end
\KKcodeE

\KKcodeS
-- 演算子と文字列操作のテスト
function KKV.process(text)
  local len = #text
  local result = "{ " .. text .. " }" -- 連結演算子
  return result
end

local my_table = { key = "value", num = 123 }
\KKcodeE

\KKcodeS+
-- KKVの内部ロジックをシミュレート
function KKV.encode(str)
  local res = str:gsub(".", function(c)
    return string.format("%%%02X", string.byte(c))
  end)
  return res
end

local output = KKV.encode("Hello KKTeX!")
print(output)
\KKcodeE
\end{document}