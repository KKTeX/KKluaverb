\documentclass{ltjsarticle}
\usepackage{luwa-ul}
\usepackage{KKluaverb}
\usepackage{tabularray, booktabs, array}
\usepackage{makeidx}

\KKvSetMap{ }{␣}
\KKvSetMap{。}{.}
\KKvSetMap{、}{,}

\begin{document}

\tableofcontents
\listoftables

\section{\KKverb|\underLineKK|との複合テスト}

\underLineKK{\KKverb|あ あ。和文の下線テスト。|}

{%
\KKvSetLineBreak{2}%
\underLineKK{いいいい\KKverb|あ あ。
改行しても「あ」が「い」になり、
句点がドットになり、スペースが可視化されるか。|あああああ}}

\section{表・複雑な構造でのテスト}

\begin{table}[h]
\caption{\KKverb|\KKluaverb|動作検証用テーブル}
\centering
\begin{tabular}{@{}>{\color{red}}l p{10cm}@{}}
\toprule
テスト項目 & 表示結果 \\ \midrule
インライン & \verb|const x = "あ あ";| \\
下線+色 & \underLineKK{\KKvOpChange{color=orange}\KKverb|orange under line|}\\
置換連鎖 & \KKverb|あ。あ。あ。|(「い.い.い.」になれば成功) \\ \bottomrule
\end{tabular}
\end{table}


\begin{tblr}{hlines, vlines, colspec={X[l]}}
  \underLineKK{\KKverb|ここでも
  繋がるかな?|} \\
  {
    次のセルとの境界テスト:
    \underLineKK{\KKverb|あああ
    いいい|}
  }
\end{tblr}

\begin{itemize}
  \item[\KKverb|\foo|] ああああ
  \item[\KKverb|\Foo%|] ああああ
\end{itemize}

\section{索引のテスト}
ここでの単語を索引に登録します:\KKverb|index_test| \index{index@\KKverb|index_test|}

句点を含む索引:\KKverb|句点。テスト| \index{くてん@\KKverb|句点。テスト|}

\underLineKK{\KKverb|下線付き索引|} \index{かせん@\underLineKK{\KKverb|下線付き索引|}}

バックスラッシュを含む:\index{めいれい@\KKverb|\foo\bar|}

\printindex 
\end{document}