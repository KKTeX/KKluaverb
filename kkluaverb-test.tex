\documentclass{ltjsarticle}
\usepackage{luwa-ul}
\usepackage{KKluaverb}
\usepackage{tabularray, booktabs, array}
\usepackage{makeidx}
\usepackage[most]{tcolorbox}

\makeatletter
\newcommand{\ascb@parindent}[1]{%
  \setlength{\parindent}{#1}\relax%
}
\def\ascb@zw#1#2{#1\zw}
\def\ascb@gtfamily{\gtfamily}
\DeclareTColorBox{simple}{ o m O{.5} O{} }% 
{empty, left=2mm, right=2mm, top=-1mm, attach boxed title to top left={xshift=\ascb@zw{1.2}{11pt}}, boxed title style={empty,left=-2mm,right=-2mm}, colframe=black, coltitle=black, coltext=black, breakable, 
before upper={\ascb@parindent{\zw}},%%%
underlay unbroken={\draw[black,line width=#3pt](title.east) -- (title.east-|frame.east) -- (frame.south east) -- (frame.south west) -- (title.west-|frame.west) -- (title.west); },
underlay first={\draw[black,line width=#3pt](title.east) -- (title.east-|frame.east) -- (frame.south east) ;
\draw[black,line width=#3pt] (frame.south west) -- (title.west-|frame.west) -- (title.west); },
underlay middle={\draw[black,line width=#3pt](frame.north east) -- (frame.south east) ;
\draw[black,line width=#3pt](frame.south west) -- (frame.north west) ;},
underlay last={\draw[black,line width=#3pt](frame.north east) -- (frame.south east) -- (frame.south west) -- (frame.north west) ;},
fonttitle=\ascb@gtfamily, IfValueTF={#1}{title=\hspace*{.1em}【#2】〈#1〉\hspace*{.1em}}{title=\hspace*{.1em}【#2】\hspace*{.1em}},#4}
\makeatother

\KKvSetMap{ }{␣}
\KKvSetMap{。}{.}
\KKvSetMap{、}{,}

\makeindex

\begin{document}

\tableofcontents
\listoftables

\section{\KKverb|\underLineKK|との複合テスト}

\underLineKK{\KKverb|あ あ。和文の下線テスト。|}

{%
\KKvSetLineBreak{2}%
\underLineKK{いいいい\KKverb|あ あ。
改行しても「あ」が「い」になり、
句点がドットになり、スペースが可視化されるか。|あああああ}}

\section{表・複雑な構造でのテスト}

\begin{table}[h]
\caption{\KKverb|\KKluaverb|動作検証用テーブル}
\centering
\begin{tabular}{@{}>{\color{red}}l p{10cm}@{}}
\toprule
テスト項目 & 表示結果 \\ \midrule
インライン & \verb|const x = "あ あ";| \\
下線+色 & \underLineKK{\KKvOpChange{color=orange}\KKverb|orange under line|}\\
置換連鎖 & \KKverb|あ。あ。あ。|(「い.い.い.」になれば成功) \\ \bottomrule
\end{tabular}
\end{table}


\begin{tblr}{hlines, vlines, colspec={X[l]}}
  \underLineKK{\KKverb|ここでも
  繋がるかな?|} \\
  {
    次のセルとの境界テスト:
    \underLineKK{\KKverb|あああ
    いいい|}
  }
\end{tblr}

\begin{itemize}
  \item[\KKverb|\foo|] ああああ
  \item[\KKverb|\Foo%|] ああああ
\end{itemize}

\section{索引のテスト}
ここでの単語を索引に登録します:\KKverb|index_test| \index{index@\KKverb|index_test|}

句点を含む索引:\KKverb|句点。テスト| \index{くてん@\KKverb|句点。テスト|}

\underLineKK{\KKverb|下線付き索引|} \index{かせん@\underLineKK{\KKverb|下線付き索引|}}

バックスラッシュを含む:\index{めいれい@\KKverb|\foo\bar|}

\section{テキストボックスのテスト}
\begin{simple}{タイトル}
  {%
  \KKvSetLineBreak{2}%
  いいいい\KKverb|あ あ。
  改行しても「あ」が「い」になり、
  句点がドットになり、スペースが可視化されるか。|%
  あああああ
  }
\end{simple}

\printindex 

\end{document}