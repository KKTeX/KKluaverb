\documentclass{ltjsarticle}
\usepackage{luwa-ul}
\usepackage{KKluaverb}
\usepackage{tabularray, booktabs, array}
\usepackage{makeidx, amsmath}
\usepackage[most]{tcolorbox}

\makeatletter
\newcommand{\ascb@parindent}[1]{\setlength{\parindent}{#1}\relax}
\def\ascb@zw#1#2{#1\zw}
\def\ascb@gtfamily{\gtfamily}
\DeclareTColorBox{simple}{ o m O{.5} O{} }{
    empty, left=2mm, right=2mm, top=-1mm, 
    attach boxed title to top left={xshift=\ascb@zw{1.2}{11pt}}, 
    boxed title style={empty,left=-2mm,right=-2mm}, 
    colframe=black, coltitle=black, coltext=black, breakable, 
    before upper={\ascb@parindent{\zw}},
    underlay unbroken={\draw[black,line width=#3pt](title.east) -- (title.east-|frame.east) -- (frame.south east) -- (frame.south west) -- (title.west-|frame.west) -- (title.west); },
    underlay first={\draw[black,line width=#3pt](title.east) -- (title.east-|frame.east) -- (frame.south east) ;
    \draw[black,line width=#3pt] (frame.south west) -- (title.west-|frame.west) -- (title.west); },
    underlay middle={\draw[black,line width=#3pt](frame.north east) -- (frame.south east) ;\draw[black,line width=#3pt](frame.south west) -- (frame.north west) ;},
    underlay last={\draw[black,line width=#3pt](frame.north east) -- (frame.south east) -- (frame.south west) -- (frame.north west) ;},
    fonttitle=\ascb@gtfamily, IfValueTF={#1}{title=\hspace*{.1em}【#2】〈#1〉\hspace*{.1em}}{title=\hspace*{.1em}【#2】\hspace*{.1em}},#4
}
\makeatother

\KKvSetMap{ }{␣}
\KKvSetMap{。}{.}
\KKvSetMap{、}{,}
\KKvSetMap{A}{[A-mapped]}
\KKvSetMap{!}{〈EXCLAMATION〉}

\makeindex

\begin{document}

\tableofcontents
\listoftables

\section{Basic Mapping \& Edge Cases}

\KKverb|Hello World!| 

\KKverb|      Leading and trailing spaces      |

\KKverb|A!A!A!A!A|

\KKverb|# $ % ^ & _ { } ~ \ |

\KKverb|!!。。、、  !!|

\section{Nesting with \texttt{luwa-ul}}

\underLineKK{Normal Text \KKverb|Verb Text with Space| Normal Text}

\underLineKK{\KKvOpChange{color=blue}\KKverb|Blue underline with A and !|}

\underLineKK{
  \KKverb|Multi-line|
  \underLineKK{\KKvOpChange{color=red}\KKverb|Nested Red Underline|}
  \KKverb|Back to Normal|
}

\section{Complex Structures (Tables \& Lists)}

\begin{table}[h]
  \caption{移動引数テスト: \KKverb|Caption \ # %|}
  \begin{tblr}{
    colspec={Q[m,c] X[l,m]}, % X列に m (middle) を追加。これで段落モードが有効になります
    hlines, vlines,
    row{1}={bg=gray9, font=\bfseries}
  }
  項 & テスト内容 \\
  1 & \KKverb|Standard Verb| \\
  2 & \underLineKK{\KKverb|Underlined Verb in Table|} \\
  3 & {\KKvSetLineBreak{2}\KKverb|Line 1
  Line 2 with space
  Line 3|} \\
  4 & \KKverb|Special Chars: { } [ ] < > ( )| \\
  \end{tblr}
\end{table}

\begin{itemize}
  \item[\KKverb|Point 1|] \KKverb|Description with 。 and 、|
  \item[\underLineKK{\KKverb|Point 2|}] \underLineKK{\KKverb|Underlined Item Label| \index{item@\KKverb|item|}}
\end{itemize}

\section{Advanced Multi-line \& Box Stress Test}

%%% Remain to be enhanced...
\begin{simple}{Multi-language Code Test}
ああああああああああああああああああああ
ああああああああああああああああああああ
ああああああああああああああああああああ
ああああああああああああああああああああ%
{\KKvSetLineBreak{2}%
\KKverb|function hello() {
print("こんにちは、世界。"); -- space check
return A + 1; !
}

-- Testing Long Sentence
This is a long sentence that should potentially wrap or be handled by the line break settings of the package. 
和文と英文が混ざった状態で。句点。と、読点、の置換。
ただの和文のテストコード。
ここにグルーが入ってズレてほしくない。
abcみたいな英字でも、
123みたいな数字でも。
|}
ああああああああああああああああああああ
ああああああああああああああああああああ
ああああああああああああああああああああ
ああああああああああああああああああああ%
\end{simple}

\section{Indexing Strategy Test}

\index{Alphabet@\KKverb|Alphabet Mapping: A|}
\index{Symbol@\KKverb|Symbol: # $ % &|}
\index{Japanese@\KKverb|和文。置換、テスト|}

\KKverb|Check the index for A, !, 。, and 、|

\section{Deep Grouping \& Scoping}

{
  \KKvSetMap{ }{[SPACE]}
  \KKvSetMap{。}{[PERIOD]}
  \KKverb|Scoped Map Test。|
}

\KKverb|Back to Global Map Test。|

\section{Extreme Math/Macro Integration}

Use \KKverb|\KKverb| in a mathmode:
\begin{equation}
  \text{Equation with Verb: } \mbox{\underLineKK{\KKverb|E = mc^2 \alpha \beta \gamma|}}
\end{equation}

Normal math mode:
\begin{equation}
  \text{Equation with Verb: } E = mc^2 \alpha \beta \gamma
\end{equation}

Foot note\footnote{Footnote Test: \KKverb|Verb in footnote #%| and \underLineKK{\KKverb|underlined|}}.

\printindex

\end{document}